\documentclass[a4paper]{article}
\usepackage{hevea}
\usepackage{color}
\pagecolor{white}

\oddsidemargin=4mm
\evensidemargin=-1mm
\topmargin=-7mm
\textwidth=15.42cm
\textheight=23.2cm

\newcommand{\gfsweb}{http://gfs.sf.net}
\newcommand{\htmladdnormallinkfoot}[2]{\footahref{#2}{#1}}
\loadcssfile{faq.css}
\renewcommand{\cuttingunit}{subsection}

\begin{document}

\mbox{}\vspace{1cm}
\begin{center}
{\huge The Gerris FAQ}\\
\vspace{5mm}
{\large St\'ephane Popinet\\
{\tt popinet@users.sf.net}\\
\vspace{5mm}
\today}
\vspace{1cm}
\end{center}

\tableofcontents

\section{General questions}

\subsubsection{What does ``Gerris'' mean?}

Gerris is the Latin (and French) name of the water strider (or water
boatman), an aquatic insect which uses surface tension to ``walk'' on
the surface of the water. Have a look at the logo on the front page of
the Gerris \htmladdnormallinkfoot{web site}{\gfsweb} for a
graphical description.

\subsubsection{How is ``Gerris'' pronounced?}

With a soft `g' like `genetics' or `general'.

\subsubsection{Where are the printable versions of the docs?}

The tutorial, FAQ and examples pages all have printable PDF (\htmladdnormallinkfoot{tutorial.pdf}{\gfsweb/tutorial/tutorial.pdf}, \htmladdnormallinkfoot{faq.pdf}{\gfsweb/faq/faq.pdf}, \htmladdnormallinkfoot{examples.pdf}{\gfsweb/examples/examples.pdf}) and Postscript (\htmladdnormallinkfoot{tutorial.ps.gz}{\gfsweb/tutorial/tutorial.ps.gz}, \htmladdnormallinkfoot{faq.ps.gz}{\gfsweb/faq/faq.ps.gz}, \htmladdnormallinkfoot{examples.ps.gz}{\gfsweb/examples/examples.ps.gz}) versions,
the reference manual only exists as an \htmladdnormallinkfoot{HTML document}{\gfsweb/reference/book1.html}.

\subsubsection{What grid generator is Gerris using?}

Gerris uses an ``embedded boundary'' technique. Grid generation reduces
to the computation of the ``shape'' (surface and volume fractions) of
Cartesian (cubic) cells cut by the solid boundaries. These ``boolean
operations'' between solids are performed automatically using GTS (the
GNU Triangulated Surface Library). The cells cut by the boundaries can
then be refined automatically using the quad/octree structure of the
discretisation. Mesh generation is entirely automatic and works for
any input geometry (provided it is topologically consistent, i.e. an
orientable non-self-intersecting manifold).

\subsubsection{Can Gerris handle unstructured tetrahedral meshes?}

Gerris uses a quadtree (octree in 3D) finite volume discretisation. It
cannot handle unstructured meshes.

\subsubsection{Are there any plans to extend Gerris to RANS (Reynolds-Averaged Navier-Stokes)?}

The focus is on time-dependent Navier-Stokes, so RANS is not really in
my mind but nothing is in the way if this is what you need.

Large Eddy Simulation (LES) models is what I am thinking about as far
as turbulence modelling is concerned.

\subsubsection{What boundary conditions are in the code now and what
BCs are planned for the near future?}

Slip, no-slip solid boundaries, inflow, outflow, periodic\dots, but
everything is there to implement your own boundary conditions as
Gerris can not supply all the boundary conditions users could think
of.

\subsubsection{How is Gerris parallelised? Does it use MPI or a shared memory technique?}

Gerris uses a domain decomposition approach using MPI for
synchronisation at domain boundaries. For the moment it does not do
dynamic balancing of domain sizes which limits its applicability to
statically refined problems.

\subsubsection{Does adaptive refinement work in parallel?}

Yes but there may be load-balancing issues. I
don't use the MPI version for the moment. For the type of studies I am
interested in, it is usually much easier to do ``direct parallelism''
i.e. run several simulations with different parameters ``in parallel''.

\subsubsection{Does the parallel version of the code do load-balancing?}

No, the code does not have parallel load-balancing capabilities at the
moment. It can do static load-balancing i.e. ``optimal'' partitioning of
the domain so that an initial mesh is divided in roughly equal-sized
subdomains while minimising the size of the communication boundaries.

I don't personally use the parallel capability of the code very
much. What I usually need is several different sequential computations
with a different set of parameters. This is of course the ideal case
for ``parallelism''. I don't usually require to run ``one shot'' very
large parallel computations.

\subsubsection{When will parallel load-balancing be available?}

Load-balancing is not very high on my list of priorities at the
moment. I don't see fundamental obstacles to dynamic
load-balancing. The main limitation of the current code which would be
hard to do away with is the fact that only ``coarse grain'' parallelism
is possible (i.e. domains can be partitioned only at the GfsBox level
not at an individual cell level). This is a limitation only when the
ratio of total number of cells to total number of CPUs becomes small
however.

What's ultimately needed to implement ``full'' load-balancing is a way
in the code to transfer entire GfsBoxes from one processor to the
other (ensuring the correct restructuring of associated boundary
conditions). This is a technical problem but which could be solved
relatively easily by someone with a good understanding of the code
structure at the GfsBox/GfsBoundary level.

An intermediate possibly easier (but not as clean) solution can be to
save the simulation and stop the code when load-balancing becomes too
bad, then do a static load-balancing step and restart a ``new''
simulation with this as initial state. I would probably first
experiment with this approach to get used to the problems involved
first and then move to the full internally-coded solution.

\subsubsection{Can Gerris handle moving/deforming solid geometries?}

Not yet but this is planned for a next phase.

\subsubsection{Can Gerris also be used for compressible fluids?}

No, but this would be possible. The existing shallow-water solver in particular
can be seen as one form of compressible flow solver.

\subsubsection{Can Gerris solve the shallow-water (Saint-Venant) equations?}

Yes, starting with version 0.6.0, although I would not consider it
ready for ``general consumption'' right now. For simple examples on how
this works have a look in the source file in {\tt test/ocean}.

\subsubsection{How can I assist you in your development effort?}

Thanks for asking. The easiest way you can help me is first by using
the code. Setting up your own test cases etc\dots And reporting
problems, either in term of usability, unexpected results etc\dots

By doing that you will certainly help me ensure that the code is as
robust as possible and you will soon find areas which need improvement
and which you might like to work on (preferably after consultation with
me so that we can coordinate our efforts).

An important point is also to remember to try to send your
questions/comments to one of the two Gerris mailing lists (gfs-users
or gfs-devel) so that other people can benefit from the exchange (I
will also more readily reply to a message on the mailing list than to
one addressed directly to me).

\section{Installation and coding}

\subsubsection{How do I install the parallel version of Gerris?}

If when running {\tt ./configure} you got lines looking like
\begin{verbatim}
checking for mpicc... yes
\end{verbatim}
then you don't have anything else
to do. Otherwise, you need to make sure that you have MPI installed
and that the {\tt mpicc} command is in your {\tt PATH}.

For running the code in parallel, you will have to wait until I have
written the next chapter of the tutorial\dots

\subsubsection{Does the Gerris {\tt configure} script support a LAM MPI implementation?}

Yes, more generally it should support any MPI implementation which
defines a working {\tt mpicc} command accessible in the {\tt PATH}.

\subsubsection{My crappy MPI installation prevents Gerris from compiling, 
how do I turn MPI support off?}

\begin{verbatim}
% ./configure --disable-mpi
\end{verbatim}

\subsubsection{Is there a Windows version of Gerris?}

Certainly not a native Windows version (Gerris relies on features
found in professional operating systems) but one should be able to
compile and install it on Windows using \htmladdnormallinkfoot{cygwin}{http://www.cygwin.com}.

Gerris also runs on MacOSX.

My personal advice would be ``why use Windows?''

\subsubsection{Are there any plans to release a more documented version of the code?}

I have chosen the ``classical'' point of view that, if the general
description of the arguments and of what the function does (given just
before the body of (almost) all the exported functions) together with
the code in the function itself does not clearly describe what the
function is doing, then this is a problem with the code itself not
with the documentation. A counter-example of that would be a very long
monolithic code described by comments every few lines.

Also, from my personal experience, working with a number of research
and commercial codes, I would consider Gerris to be fairly well documented.
The code is also quite modular, so you shouldn't (hopefully) need to
go through all the 16000 lines of code...

Of course, I would be glad to address any detailed problem you may
have (unclear documentation etc\dots)

\subsubsection{Are there any plans to release a C++/Java/Object Oriented 
implementation of Gerris?}

No. Gerris is already object-oriented (with class inheritance etc\dots),
see the tutorial for an example of how this works.

\section{Physics and dimensioning}

\subsubsection{Where are variables like viscosity, density etc\dots defined?}

By default, the density is unity and the molecular viscosity is
zero (i.e. there is no explicit viscous term in the momentum
equation). In practice, it does not mean that there is no viscosity at 
all however, because any discretisation scheme always has
some numerical viscosity. Of course, the lower the numerical
viscosity, the better. Gerris has quite good properties in this
respect.

\subsubsection{How come Gerris generates a Von Karman vortex street for 
an inviscid flow around a half-cylinder? I would expect the inviscid flow 
to remain irrotational.}

This is perfectly right in the case of flow around smooth
solid boundaries. If there is a sharp corner (as for the
half-cylinder), the potential flow solution is singular in the sense
that the velocity tends to infinity as one gets closer to the
corner. In practice (finite difference numerical solution) and in
reality, the local numerical (or real) viscosity near the corner,
smears out the singularity, which results in the creation of a (point) 
source of vorticity which is then carried away by the mean flow (as
you can see on the half-cylinder example).

Even in the case of a smooth geometry, numerical inaccuracies in the
boundary conditions on the solid surface can lead to the generation of 
a small amount of vorticity (much smaller than what is generated at a
discontinuity though).

\subsubsection{How would I create a $5\times 5$ box?}

It is possible to change the size of the unit GfsBox, however, I would
encourage you to think in ``relative units'' rather than ``absolute
units''. When studying fluid mechanics (and other physical) problems it
is almost always a good idea to use non-dimensional units. This makes
relevant independent parameters (such as the Reynolds number for
example) immediately apparent. When using Gerris I would recommend
scaling all your physical input parameters by a reference length (the
physical length of the GfsBox). This also eliminates the need for
changing the length of the GfsBox.

\subsubsection{How would I modify the file you sent me ({\tt tangaroa.gfs}) for a ship that is 150 
meters long and exposed to a cross-flow wind velocity of 50 meters/sec?}
 
You would have to non-dimensionalise both the model ship geometry and wind speed.

The reference length of the {\tt GfsBox} would be $3*150\;meters$, so you would
scale the model geometry by a factor of $1/(reference\;length)$ or $1/450$.

You might want to use the {\tt transform} program to do
that, something like this:
\begin{verbatim}
% transform --scale 2.22222e-3 < model.gts > model_scaled.gts
\end{verbatim}

You also have to keep in mind that the bottom boundary of a 3D box is
at $z = -0.5$. You want to have that coinciding with the sea level
(i.e. translate your model vertically by the correct amount).

\subsubsection{How would I redimensionalise U,V,W and P?}

$$U\;meters/sec = U*Uref = U*50\;meters/sec$$
$$V\;meters/sec = V*Uref = V*50\;meters/sec$$
$$W\;meters/sec = W*Uref = W*50\;meters/sec$$
$$P\;Pascals = P*DENSITY*Uref^2$$

However, keep in mind that the only relevant parameter for the
(constant density) Navier-Stokes equations is the Reynolds number. If
you do not include any explicit viscous term the (theoretical)
Reynolds number is always infinite. In practice this means that the
inflow velocity has only a uniform scaling influence on the final
solution. For example
\begin{description}
\item[simulation 1:] inflow velocity set to 1.0
\item[simulation 2:] inflow velocity set to 2.0
\end{description}
then, the velocity field of simulation 1 at time t is exactly equal
(to machine precision) to the velocity field for simulation 2 at time
$t/2.0$, divided by 2.0.

\subsubsection{It looks like t and dt output by GfsOutputTime are also scaled?  How would I scale t and dt to time in seconds?}

Let's say your reference scale is $L=450\;meters$, your reference speed
$U=50\;meters/sec$, your reference time is then $T=L/U=9\;sec$

You thus need to multiply both t and dt by $T=9\;sec$.

\subsubsection{How do I scale Vorticity?}

The units of vorticity are 
$$LT^{-1}/L \rightarrow T^{-1}$$
and
$$ T' = T*Lref/Vref$$
therefore
$$
VORTICITY' = VORTICITY*Vref/Lref
$$

\subsubsection{The code provides support for the variable density incompressible Euler 
equations.  Does that mean you can input the density of the fluid density 
(air, water, etc\dots)?}

Not really if what you mean is a constant density throughout the
domain. In the case of the incompressible constant-density Navier-Stokes
equations, the density is irrelevant. It is only a scaling factor for
the pressure.

What this really means is that Gerris can deal with flows where
the density varies across the domain (e.g. a mixture of two miscible
fluids, or density variations due to salinity variations in the sea for
example).

\subsubsection{Although the initialised problem is symmetric, the solution 
becomes asymmetric as time passes, why?}

The code is indeed not perfectly numerically symmetrical. This is due
mainly to the tolerance in the solution for the pressure equation, if
you decrease the tolerance you should see smaller
asymmetries. You can do this using
\begin{verbatim}
  ApproxProjectionParams {
    tolerance = 1e-6
  }
  ProjectionParams {
    tolerance = 1e-6
  }
\end{verbatim}

\subsubsection{How do I deal with negative values of the pressure?}

Your question is interesting, it comes down to the meaning of
``pressure'' for incompressible flows.

For {\em compressible} flows ``pressure'' has a thermodynamic definition
and is directly linked to other physical quantities through an
equation of state. It is defined on an absolute scale.

For {\em incompressible} flows ``pressure'' does not have a thermodynamic
definition (there is no equation of state linking it to other physical
quantities), rather it comes about as the stress field necessary to
enforce the incompressibility condition. In this context, only its
gradients are relevant, not its absolute value i.e. one can add any
constant to the pressure field without changing the solution.

Conclusion: If you don't like negative pressures just add any constant
necessary to make them positive.

\section{Representation of solid boundaries}

\subsubsection{How do I import my geometry into Gerris?}

You need to convert your geometry into a set of triangulated surfaces
and be able to export it in the GTS format (very simple, described
\htmladdnormallinkfoot{here}{http://gts.sourceforge.net/reference/gts-surfaces.html\#GTS-SURFACE-WRITE})
or alternatively in the STL format (which can be converted to GTS
using the {\tt stl2gts} program).

The tricky bit is that the surfaces you export must represent proper
solid objects i.e. they must be orientable, closed, manifold and non
self-intersecting surfaces.

\subsubsection{What CAD package can I use to export STL/GTS files?}

Blender can do that and is open-source, also have a look at ac3d, k3d
and Pro/Engineer, Rhino. There are plenty of others.

\subsubsection{Do I need to tessellate (increase the number of triangles of) 
my surface before importing it into GTS?}

If your solid boundary is exactly defined using a few triangles, there
is no need to use more. In short, the mesh size generated by Gerris is
completely independent from the ``triangle size'' of the input surface
(in contrast to what happens in ``classical'' unstructured mesh
solvers).

If for example, you want to resolve the boundary layers around your
solid, you could tell Gerris to use a ``fine enough'' mesh like this:
\begin{verbatim}
GfsRefineSolid 10
\end{verbatim}
which tells Gerris to use 10 levels of refinement near the solid
surface. ``Fine enough'' is going to depend on the details of the
physics (most importantly Reynolds number) and on the constraints in
term of computational time, memory size etc\dots

\subsubsection{Which part of the parameter file tells Gerris where the
 half-cylinder is placed? How do I alter it?}

The position of the solid object is defined (obviously) through the
coordinates of its vertices. If you created it using a CAD or similar
program, you can translate, rotate etc\dots the object using this same
program.

Alternatively, you can use the {\tt transform} program which comes with
GTS.
\begin{verbatim}
% transform -h
\end{verbatim}
will give you a summary of the transformations you can make, currently
\begin{verbatim}
Usage: transform [OPTION] < file.gts
Apply geometric transformations to the input.

  -r ANGLE  --rx=ANGLE      rotate around x-axis
  -m ANGLE  --ry=ANGLE      rotate around y-axis
  -n ANGLE  --rz=ANGLE      rotate around z-axis
  -s FACTOR --scale=FACTOR  scale by FACTOR
  -R FACTOR --sx=FACTOR     scale x-axis by FACTOR
  -M FACTOR --sy=FACTOR     scale y-axis by FACTOR
  -N FACTOR --sz=FACTOR     scale z-axis by FACTOR
  -t V      --tx=V          translate of V along x-axis
  -u V      --ty=V          translate of V along y-axis
  -w V      --tz=V          translate of V along z-axis
  -i        --revert        turn surface inside out
  -o        --normalize     fit the resulting surface in a cube of
                            size 1 centered at the origin
  -v        --verbose       print statistics about the surface
  -h        --help          display this help and exit

Reports bugs to popinet@users.sourceforge.net
\end{verbatim}
The resulting (transformed) object is written on the standard output.

For example, if you want the half-cylinder in the second cell do:
\begin{verbatim}
% transform -t 1 < half-cylinder.gts > half-cylinder1.gts
\end{verbatim}
and use {\tt half-cylinder1.gts} in the parameter file.

\subsubsection{Are there any tools for converting format-X (not STL) files 
(generated via a CAD system) to a GTS-format file?}

The GTS file format is described \htmladdnormallinkfoot{here}{http://gts.sourceforge.net/reference/gts-surfaces.html\#GTS-SURFACE-WRITE}.
It is very simple. You should be able to write your own filter
using your favourite scripting language. You might want to have a look 
at the {\tt cleanup} utility which comes with GTS (in the {\tt examples/} 
directory) It will allow you to link unlinked faces, remove duplicate
vertices etc\dots

\subsubsection{Gerris seem to allow only one solid body, is this correct?}

No, there is no limitation on the number and/or complexity of solid
bodies (as long as they are properly oriented, manifold geometrical
surfaces). Multiple bodies are possible, either as a single GTS file
containing multiple separate bodies or as multiple calls to
{\tt GtsSurface} in the parameter file with several non-intersecting GTS
surfaces.
\begin{verbatim}
GtsSurfaceFile box_1.gts
GtsSurfaceFile box_2.gts
\end{verbatim}
Note however that the solids cannot intersect.

\subsubsection{How do I orient my solid surfaces properly?}

The orientation of the
faces of your solid defines where the fluid side is (by
convention the counter-clockwise (CCW) normal direction to a face points toward the solid
side). If your solid is not oriented properly you can use the
{\tt --revert} or {\tt -i} option of {\tt transform} to turn it ``inside out''.

\subsubsection{It looks like all STL files need to be turned ``inside out''.  I don't understand why, but {\tt transform -i} fixed the problem.}

It is just a matter of different conventions. The program you use has
chosen to orient the CCW face normals toward the ``outside'' of the
solid object.

\subsubsection{Can solids intersect?}

No, you first need to use the ``boolean operations'' or ``constructive
solid geometry'' operations of your solid modeller to generate the
union of your solids.

This may change in the future.

\subsubsection{We have a problem inserting some GTS files generated 
from STL files and even inserting the standard GTS files found 
on the GTS samples site?}

The samples files on the GTS site are not
necessarily describing consistent geometric surfaces (i.e. they can be
open, non-manifold etc\dots)

\section{Post-processing and Visualisation}

\subsubsection{Is there a way to view the results and solid(s) at the same time (with the 
solid in the correct location? (Geomview question)}

Yes, you need to select the Inspect$\rightarrow$Appearance$\rightarrow$Normalise$\rightarrow$None
option in the Geomview menu. The default is to ``normalise''
(i.e. rescale) each object individually (Normalise$\rightarrow$Individual option)
so that it fits at the centre of the viewed area.

\subsubsection{Is there any way to output U, V, W, P, etc\dots at point (X, Y, Z) in the flow field?}

Yes, use
\begin{verbatim}
GfsOutputLocation { step = 1 } data -0.209371 -0.0166124 -0.449834
\end{verbatim}
where the last three numbers are the $x,y,z$ coordinates of the location
where you want the values of the variables or
\begin{verbatim}
GfsOutputLocation { step = 1 } data positions
\end{verbatim}
where {\tt positions} is a file containing newline-separated coordinates.

The file {\tt data} will contain the data as described in the first line of the file, a ``comment'' line starting with \#.

\subsubsection{Where is the description of the format of the data 
section of saved simulation files?}

If you intend to read the simulation files (to convert them or do
other operations/calculations etc\dots) I would highly recommend that
you do so through the functions provided by the Gerris library
({\tt gfs\_simulation\_read()} and so on). This way you will not reinvent the
wheel and you will be able to use all the functionalities provided by
the library (traversal of the octree structure, computation of
gradients, interpolations etc\dots). This would also ensure that your
code is independent of the format changes in the simulation file.

Just to give you an example on how this can bite you:

the GfsOutputSimulation object can be used like this (in simulation
files)
\begin{verbatim}
  GfsOutputSimulation { step = 0.1 } sim-%3.1f { variables = P,C }
\end{verbatim}
in this case, the simulations files ({\tt sim-0.1}, {\tt sim-0.2} etc\dots) will
only contain the {\tt P} and {\tt C} variables.

Your code which reads the simulation files would need to know about
this. The {\tt gfs\_simulation\_read()} function deals with that for you and
other functions give you easy access to this kind of information (what
variables where contained in the simulation file etc\dots)

\subsubsection{How do I compute/display the vorticity field with OpenDX?}

The {\tt DivCurl} tool in OpenDX works for general grids. You can use it
to compute the vorticity (vector) from the {\tt U} field returned by
{\tt GfsImport}, however this will work only for 3D fields.

\subsubsection{The GfsOutputSimulation and GfsOutputLocation files only place up
to eight decimals. Is there any way to increase the number of decimal places?}

Good question. No there isn't, short of editing and recompiling the
source code. That would be a nice option to have in the simulation
file.

\subsubsection{I would like a time-averaged velocity profile,
Would I have to specify a number of monitoring points at different heights,  
or is there a method to time average over a line through the solution 
domain?}

There is no method to do line averaging at the moment, however there
is a method which averages (or more exactly stores the sum) of a given
variable in time over the whole domain. You can do it like that:
\begin{verbatim} 
  EventSum { start = 1 istep = 1 } U SUx
  EventSum { start = 1 istep = 1 } V SUy
  EventSum { start = 1 istep = 1 } W SUz
  EventSum { start = 1 istep = 1 } U*U SU2x
  EventSum { start = 1 istep = 1 } V*V SU2y
  EventSum { start = 1 istep = 1 } W*W SU2z
  OutputSimulation { start = end } simulation-sum { 
    variables = SUx,SUy,SUz,SU2x,SU2y,SU2z
  }
\end{verbatim}
which would add {\tt U} to {\tt SUx} at every timestep ({\tt istep = 1}) starting from
time 1 ({\tt start = 1}) etc\dots and {\tt U*U} to {\tt SU2x} at every timestep
etc\dots The resulting sums are then written at the end of the
simulation in the file {\tt simulation-sum}. This file can then be
post-processed (using {\tt gfs2oogl} for example) to obtain averages,
standard deviations etc\dots (along any curves you want of course).

\subsubsection{Using animate, the sequence of images generated by OutputPPM looks weird, what's happening?}

This is probably an artefact of the way the {\tt animate} command
displays a series of PPM images. What happens is that {\tt OutputPPM}
generates PPM images which are just big enough to contain all the data
in your simulation e.g. if you use 7 levels of refinement and one box,
{\tt OutputPPM} will generate images with $128\times 128$ pixels. If
you use an adaptive resolution with a maximum level of 6, the size of
the resulting image generated by {\tt OutputPPM} can be anything in
$1\times 1$, $2\times 2$, $4\times 4$, $8\times 8$, $16\times
16$, $32\times 32$, $64\times 64$ depending on the maximum number of
levels necessary to verify your adaptation criterion. As a result,
{\tt animate} can see a series of PPM images with a variable size, if
you look carefully you will see that the weird patterns you see are
smaller-size images of your simulation, displayed in the top-left
corner of the initial image. What {\tt animate} should really do is blank
out the previous larger image before displaying the smaller image, to
make the difference in size clear.

The solution is simple, you can set the size of the images generated by {\tt OutputPPM} using:
\begin{verbatim}
OutputPPM { step = 0.05 } tracer.ppm { v = T maxlevel = 6 }
\end{verbatim}
which will result in PPM images of size $64\times 64$, independently
of the maximum level of refinement in the simulation.

\subsubsection{Why create a new visualisation tool like 
GfsView? Can't you use existing tools like Mayavi/VTK, OpenDX etc\dots?}

Most visualisation packages assume that the data is defined
on either structured Cartesian meshes (this includes curvilinear
coordinates) or fully unstructured meshes (tetrahedra etc\dots). 

The octrees used by Gerris need first to be converted into
unstructured tetrahedra and then imported into OpenDX etc\dots This is
quite slow and memory-hungry and loses most of the advantages of
the octree: in particular the multilevel representation of the
solution is very useful from a visualisation point of view.

I am not aware of any good visualisation tool which understands
octrees. It would be a good idea to post messages on OpenDX, Mayavi, VTK
etc\dots mailing lists asking about support for octrees. I did that and
got little feed back, but more messages would show the developers of these
projects that there is a desire for such a feature.

GfsView makes the most of the octree structure to accelerate
visualisation, computation of isosurfaces etc\dots

\section{Running Gerris}

\subsubsection{Are the files {\tt vorticity.gfs} and {\tt half-cylinder.gfs} included in the 
Gerris distribution?}

No, you will need to type them following the tutorial\dots

\subsubsection{Your flow analysis for the RV Tangaroa is just the type of problem I would 
like to be able to solve quickly, etc\dots  How long did it take to setup and run 
this problem?  Could you send me a copy of the input file?}

The longest was to get a ``proper'' CAD model of the vessel. We had it
made by the ship designers but it was full of topological
inconsistencies (folds, degenerate faces etc\dots). It was a real pain
to fix it. Once you have a proper orientable, manifold solid there is
nothing more to do really.

Here is a parameter file (replace {\tt tangaroa.gts} with your model)
\begin{verbatim}
1 0 GfsSimulation GfsBox GfsGEdge {} {
  Time { end = 3 }
  GtsSurfaceFile tangaroa.gts
  Refine 5
  RefineSolid 9
  Init {} { U = 1. }
  AdaptVorticity { istep = 1 } { maxlevel = 8 cmax = 1e-2 }
  OutputSolidStats {} stdout
  OutputTime { istep = 1 } stdout
  OutputBalance { istep = 1 } stdout
  OutputProjectionStats { istep = 1 } stdout
  OutputSimulation { start = 0.1 step = 0.1 } simulation-%03.1f {
	variables = U,V,W,P 
  }
  OutputTiming { start = end } stdout
}
GfsBox { right = BoundaryOutflow left = BoundaryInflowConstant 1. }
\end{verbatim}
It took about 12 hours (and 100 MB RAM) on a single CPU 350 MHz
Pentium (knowing that the maximum length of the model I use is about a
third of the domain size).

\subsubsection{How do you modify the mesh to give greater detail in
a specific area of flow?}

As you saw in the tutorial, the meshing is automatic and follows
user-defined criteria. Vorticity and gradient-based criteria for
example, can be used. The level of refinement used for both the
initial refinement and the adaptive refinement can both be functions
of space, time, other variables etc\dots which give almost total
flexibility. Other criteria can be added within the object-oriented
framework of the code if necessary.

\subsubsection{Is there any way to initialise the grid so that a fine grid is generated 
around the surface of the solid and a coarser grid is generated in the flow 
field (resulting in significantly fewer cells being generated)?}

Sure. You can use something like:
\begin{verbatim}
GfsRefine 6
GfsRefineSolid 8
\end{verbatim}
which will first create a uniform level 6 grid and then add two extra
levels (up to level 8) only in the cells cut by the solid boundary.

\subsubsection{Is it possible to turn off  adaptive meshing in defined areas?}

You could do this like that for example:
\begin{verbatim}
AdaptVorticity { istep = 1 } { 
  cmax = 1e-2 
  levelmin = (x > 0 ? 6 : 0)
  levelmax = (x > 0 ? 6 : 7)
}
\end{verbatim}
which would use a constant resolution (6 levels) on the right side of
the domain ({\tt x > 0}) and adaptive resolution (up to 7 levels) on
the left side.

\subsubsection{Is there a way to control the maximum size of a simulation?}

Use something like:
\begin{verbatim}
AdaptVorticity { maxlevel = 10 maxcells = 400000 cmax = 1e-2 }
\end{verbatim}
the {\tt maxcells} option tells the adaptive algorithm to use a maximum of
400,000 cells to discretise the domain. When this maximum number is
reached the algorithm minimises the maximum cost of the refinement by
optimally distributing the cells across the domain. You have to be
aware however that this means that the accuracy of the simulation will
not be constant in time.

\subsubsection{My simulation file looks fine but does not work, why?}

Are you sure your text editor does not
include special characters in your files? (the infamous DOS line ending
comes to mind).

\subsubsection{How do I make all the boundaries of a 2D box no-slip?}

\begin{verbatim}
GfsBox {
   top =    Boundary { BcDirichlet U 0 }
   bottom = Boundary { BcDirichlet U 0 }
   left =   Boundary { BcDirichlet V 0 }
   right =  Boundary { BcDirichlet V 0 }
}
\end{verbatim}

\subsubsection{How do I make boundary conditions time-dependent?}

What about \htmladdnormallinkfoot{this page}{http://gfs.sourceforge.net/tutorial/tutorial/node15.html} of the tutorial?
As \htmladdnormallinkfoot{described}{http://gfs.sourceforge.net/reference/gfs-functions.html} in the reference manual
functions can depend on time (just use variable {\tt t}). You should
be able to use this to implement your boundary conditions.

\subsubsection{How do I run Gerris in parallel?}

The principle is relatively simple. Each {\tt GfsBox} can take a {\tt pid}
argument which defines the number of the process on which the solution
for this GfsBox will be computed. If you take the ``half cylinder''
example and do something like:
\begin{verbatim}
4 3 GfsSimulation GfsBox GfsGEdge {} {
  Time { end = 10 }
  Refine 6
  GtsSurfaceFile half-cylinder.gts
  Init {} { U = 1 }
  OutputProjectionStats { step = 0.02 } stderr
  OutputSimulation { step = 1 } simulation-%d-%3.1f {}
  OutputTiming { start = end } stderr
}
GfsBox { pid = 0 left = BoundaryInflowConstant 1 }
GfsBox { pid = 1 }
GfsBox { pid = 2 }
GfsBox { pid = 3 right = BoundaryOutflow }
1 2 right
2 3 right
3 4 right
\end{verbatim}
if you run this using
\begin{verbatim}
% gerris2D half-cylinder.gfs
\end{verbatim}
it will run on one processor. If you now do
\begin{verbatim}
% mpirun -np 4 gerris2D half-cylinder.gfs
\end{verbatim}
it will run on 4 processor with each of the {\tt GfsBoxes} assigned to a
different processor. Gerris takes care of the communications necessary
at the boundaries between {\tt GfsBoxes} on different processors.

The data will be written in different files e.g:
\begin{verbatim}
vorticity-0.ppm, vorticity-1.ppm etc\dots
\end{verbatim}
where the number is the pid of the processor.

For a more complete description you will have to wait until I find the
time to write the corresponding section in the tutorial.

\end{document}

% LocalWords:  isosurfaces
